\\\hline
\(CT\)          &   Cicle time: tempo di permanenza di un job nel sistema\\ \hline
\(JOB\)         &   Unitá da produrre, non è importante capire se si tratta di un prodotto singolo o più prodotti\\ \hline
\(T_s(i)\)      &   Tempo di servizio o tempo di processo dello stadio \(i\), dato che è sempre affetto da variabilitá sará una \textit{variabile casuale}\\ \hline
\(E(T_s(i))\)   &   Si tratta di una media perchè \(T_s(i)\) è una \textit{variabile casuale}, quindi si considera la media\\ \hline
\(CT(i)\)       &   Cicle Time di un singolo stadio \(i\), dato che il tempo di sistema è una \textit{Variabile Casuale} allora anche il Cicle Time sará una \textit{variabile casuale}\\ \hline
\(CT_s\)        &   Cicle Time dell'intero sistema: è una funzione dei \(CT(i)\), è la somma dei singoli \(CT(i)\) solo se si tratta di un sistema lineare\\ \hline
\(CT_q(i)\)     &   Tempo di attesa o Queue time
                    \begin{figure}[H]
                        \centering
                        \includegraphics[width=.4\linewidth]{images/chapter1/3.jpg}
                        \label{fig:coda}
                    \end{figure}
                    \\ \hline
\(WIP\)         &   Misura il numero di Job presenti nel sistema in un determinato istante \\ \hline
\(WIP_s\)       &   Misura il numero di Job nel sistema in un determinato istante se si considera il sistema intero\\ \hline
\(WIP_i\)       &   Misura il numero di Job nel sistema in un determinato istante se si considera lo stadio \(i\)\\ \hline
\(TH\)          &   Throughput, è il tasso di produzione del sistema, è legato alla velocitá della macchina, non è infatti l'inverso del tempo ma più correttamente il numero di Job che escono dal sistema nell'unitá di tempo (i.e. ogni giorno, ogni mese, ogni ora etc.)
\\ \hline
\(\lambda\)     &   Corrisponde spesso al \(TH\) del sistema ma vale per l'ingresso, rappresenta il numero di Job che entrano nel sistema nell'unitá di tempo, quando ho degli split e dei merge semplicemente sommo o divido per una certa percentuale, verrá praticamente sempre assunto che tutto ciò che entra nel sistema deve poi anche uscire quindi spesso \(\lambda\)=\(TH\)\\ \hline
\(\mu\)         &   Tasso di servizio di un servente: è la quantita massima di job che il server riesce a servire nell'unitá di tempo\\ \hline
\(\frac{1}{\lambda}\)   &   Tempo di interarrivo, dato che \(\lambda\) è un tasso, per trovare il tempo occorre calcolarne il reciproco \\ \hline
\(\frac{1}{\mu}\)   &   Tempo di servizio, dato che \(\mu\) è un tasso, per trovare il tempo occorre calcolarne il reciproco \\ \hline
\(WK\)          &   Workstation / stazione / servente: il contenitore di uno o più servernti/macchine
\begin{figure}[H]
    \centering
    \includegraphics[width=.4\linewidth]{images/chapter1/5.jpg}
    \label{fig:workstation}
\end{figure}
\\ \hline
\(Step\)        &   Coppia di workstation e tempo di processo di quella workstation. 
                    \begin{itemize}
                        \item step(M1,5min)
                        \item step(M2,10min)
                        \item step(M3,8min)
                        \item step(M2,2min)
                    \end{itemize}
                    In questo caso ci sono 4 step ma le workstation sono 3 (M1, M2, M3), sono quindi presenti dei flussi rientranti, non c'è corrispondenza biunivoca tra workstation e step. \textit{N.b.} diversi Job sono dello stesso tipo solo se hanno gli stessi Step, ovvero se hanno lo stesso routing\\ \hline
\(Routing\) &   Sequenza di step \\ \hline
\(Layout\)  &   Tutti i Job che passano nel sistema determinano il Layout dello stesso \\ \hline
\(A(t)\)    &   Numero di Job entranti fino al tempo \(t\)\\ \hline
\(D(t)\)    &   Numero di Job usciti dal sistema fino al tempo \(t\)\\ \hline
\(P_n\)     &   \(Prob{N=n}\) ovvero la probabilitá di avere un numero di job nel sistema pari a \(n\)\\ \hline
\(u\)   	&   \(\frac{\lambda}{mu}\), rappresenta l'utilizzo
