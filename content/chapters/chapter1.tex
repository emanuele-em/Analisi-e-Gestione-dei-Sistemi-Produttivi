\chapter{Rappresentazione del prodotto, connessione con la struttura fisica organizzativa del sistema}
\label{sec:capitolo1}

\section{Tipologie di analisi e di sistemi}

Ci sono due tipologie di metodi di studio principali:
\begin{itemize}
    \item Analitici
    \item Di simulazione
\end{itemize}

Tutti i metodi che vedremo hanno una formula, quindi hanno un parametro po più in entrata e in uscita fornisocono un risultato, questo approccio è un approccio \textit{Analitico}.

Quando il sistema diventa molto complesso gli errori tendono ad essere non più trascurabili, in questo caso occorre utilizzare le \textit{simulazioni}

In tutte le casistiche occorre mantenere la complessitá il più basso possibile, essendo un processo automatico, il limite è quello della complessitá.
Occorre mettere in atto solamente gil elementi rilevati, senza distorsioni altrimenti lo studio non avrebbe valore.
Verrá effettuato lo studio di sistemi come:
\begin{itemize}
    \item Lineari
    \item Split
    \item Merge
    \item Rietramenti
    \item Uscenti
\end{itemize}

\begin{figure}[H]
    \centering
    \includegraphics[width=.7\linewidth]{images/chapter1/1.jpg}
    \caption{Tipologie di sistemi}
    \label{fig:tipologie_di_sistemi}
\end{figure}

Verranno effettuati studi con un approccio di scomposizione, nonostante il Network di processi verrá sempre e solo analizzato un processo per volta e verranno infine uniti con tecniche specifiche.

\textit{Stadio}: uno stadio è una composizione del processo che potrebbe essere a singola macchina o \textit{Server} o a macchine parallele, comprende anche la coda in ingresso o \textit{Buffer}.

\begin{figure}[H]
    \centering
    \includegraphics[width=.7\linewidth]{images/chapter1/2.jpg}
    \caption{Stadio}
    \label{fig:Stadio}
\end{figure}

Il sistema di produzione è composto da server + buffer, non è interessante comprendere cosa accade prima e non ci interessa cosa succede dopo.

\begin{equation}
    \begin{split}
        CT_s    &= f(CT(i)) \\
        CT(i)   &= CT_q(i) + E(T_s(i))
    \end{split}
\end{equation}

\(CT\) e \(WIP\) sono 2 misure di performance, \(CT\) mi dice quanto tempo sta nel sistema quindi quanto riesco a essere efficiente a livello di tempo, deve essere il più basso possibile, il \(WIP\) deve essere basso perché il \(WIP\) è una misura di quanto è immobilizzato, dato che si vuole vendere il prima possibile, per farlo occorre che sia processato il prima possibile, detenere è molto costoso per me. Dato che \(TH\) è il tasso di uscita, deve essere il più alto possibile. Le 3 grandezze sono legate indissolubilmente, sapendone 2 si può trovare la terza.

\subsubsection{Diversi tipi di Layout di sistema}

\begin{itemize}
    \item Se i diversi tipi di job hanno una sequenza uguale di serventi per essere prodotti ma con tempi di servizio per ogni servente diverso ho una configurazione o un layout \textit{Linea}
    \item Se ho poche varianti di Job si ha un \textit{Flowline} oppure un \textit{Flowshop}
    \item Se si hanno molte varianti si è in presenza di un layout \textit{Jobshop} o \textit{a reparti}
\end{itemize}

\subsection{Little Law}

\(WIP\), \(CT\), \(TH\) sono dipendenti dal tempo, in momenti diversi si hanno quindi valori diversi, sarebbe più corretto scrivere \(WIP(t)\), \(CT(t)\), \(TH(t)\).
Per non avere complessitá estrema si studia la situazione nel cosiddetto \textit{stady state} che tradotto significa stato di equilibrio. Il sistema non raggiunge l'equilibrio se non è un sistema \(stazionario\), si osservi la figura:

\begin{figure}[H]
    \centering
    \includegraphics[width=.9\linewidth]{images/chapter1/6.jpg}
    \caption{rappresentazione dello Stady state}
    \label{fig:stadystate}
\end{figure}

La linea blu nella figura \ref{fig:stadystate} rappresenta la realtá, il \(WIP\), come le altre grandezze continueranno sempre a variare, quella che si stabilizza è la loro media. in \(t_1\) e \(t_2\) sono diversi istantaneamente ma hanno la medesima media.

\begin{conditions*}
    WIP, CT, TH     &   determinabili da dati e grandezze osservabili, da tempi di arrivo e da tempi di uscita\\
    A(t)    &   numero di Job giá arrivati fino all'istante \(t\)\\
    D(t)    &   numero di Job già usciti fino al tempo \(t\)
\end{conditions*}

\begin{figure}[H]
    \centering
    \includegraphics[width=.7\linewidth]{images/chapter1/7.jpg}
    \caption{Curva di \(\frac{A(t)}{D(t)}\)}
    \label{fig:curvaAtDt}
\end{figure}

\begin{equation}
    \begin{split}
        CT_i    &= T_D(i)-T_A(i) \forall i \\
        WIP(t)  &= A(t) - D(t) \forall t
    \end{split}
\end{equation}

Occorro però le medie dei valori, quindi per \(CT\):

\begin{equation}
        CT = \frac{\sum_{i = 1}^{N_{ab}} [T_D(i)-T_A(i)]}{N_{ab}} 
\end{equation}
dove \(N_{ab}\) rappresenta il numero di Job arrivati o usciti nel sistema dell'intervallo scelto, \(a\) e \(b\) vengono scelti in modo che il sistema sia vuoto prima dell'istante \(a\) e torni vuoto dopo l'istante \(b\).

Mentre per \(WIP\):
\begin{equation}\label{eq:wipintegrale}
    \begin{split}
        WIP  &= \frac{\int_{a}^{b}  [A(t) - D(t)]\,dt }{b-a} \\
        \int_{a}^{b}  [A(t) - D(t)]\,dt &= CT*N_{ab}
    \end{split}
\end{equation}

\begin{figure}[H]
    \centering
    \includegraphics[width=.7\linewidth]{images/chapter1/8.jpg}
    \caption{Rappresentazione grafica dell'integrale}
    \label{fig:integralegrafico}
\end{figure}

Il secondo integrale dell'equazione \ref{eq:wipintegrale} è valido perchè è possibile vedere l'integrale stesso come somma dei singoli rettangoli in figura \ref{fig:integralegrafico}, dove ogni singolo rettangolo è \(CT_1\) (primo rettangolo in basso), \(CT_2\) (secondo rettangolo) etc.

\[
    \underbrace{\underbrace{CT_1}_{base} * \underbrace{1}_{altezza}}_{rett. 1} + \underbrace{CT_2 * 2}_{rett. 2} + \ldots
\]

è possibile quindi riscrivere il \(WIP\) come segue:
\begin{align*}
    WIP &= \frac{CT*N_{ab}}{b-a}\\
        &= CT*\underbrace{\frac{N_{ab}}{b-a}}_{TH=\lambda}
\end{align*}

\subsubsection{Legge di Little}

Ci ricaviamo quindi la legge di Little:

\begin{equation}\label{eq:littleLaw}
    \begin{split}
        WIP &= CT*TH\\
        CT  &= \frac{WIP}{TH}\\
        TH  &= \frac{WIP}{CT}
    \end{split}
\end{equation}
La legge di Little vale sempre per i valori medi, se \(TH\) è fissato e aumenta \(CT\) allora anche \(WIP\) deve aumentare, se si riesce ad abbassare il \(WIP\) allora diminuisce anche il \(TH\), a meno che non decresca \(CT\)

Avere quindi un \(TH\) molto elevato non implica avere un \(WIP\) molto elevato, in questo caso anche \(CT\) dovrebbe essere elevato.

Con i modelli di tipo analitico si cerca di capire all'equilibrio qual è la distribuzione di probabilitá che nel sistema ci sia 1 Job, 2 Job, etc...
Tornando alla curva \ref{fig:integralegrafico} possiamo visualizzare il \(TH\) graficamente:
\begin{figure}[H]
    \centering
    \includegraphics[width=.7\linewidth]{images/chapter1/th.jpg}
    \caption{Visualizzazione grafica di TH}
    \label{fig:thGrafico}
\end{figure}

il \(TH\) è la pendenza della retta che unisce il numero di Job processati all'istante \(b\) al numero di Job processati all'istante \(a\) iniziale.

\subsubsection{Ricavare le probabilitá di equilibrio per il calcolo delle grandezze studiate}

Si immagina un singolo stadio con un singolo servente e un buffer \(\infty\) con \(\lambda\) corrispondente al tasso di arrivo e \(\mu\) corrispondente al tasso di servizio\footnote{Solitamente si ha quasi paradossalmente \(\lambda = TH\) e non \(\mu = TH\)}, dato che si tratta di tassi occorre convertirli in tempi tramite i reciproci, si ha quindi che \(\frac{1}{\lambda}\) corrisponde al tempo di interarrivo mentre \(\frac{1}{\mu}\) corrisponde al tempo di servizio, entrambi i tempi sono distribuiti esponenzialemnte (\(\sim (exp)\)): la distribuzione esponenziale è fondamentale perchè si tratta di una distribuzione senza memoria, come i sistemi descritti.

\(P_n=Prob{N=n}\), sapere se nel sistema ci siano \(n\) Job è utile perché il \(\lambda\) è uguale al \(TH\), essendo \(TH\) noto, per trovare il \(WIP\) è sufficiente conoscere il valore rimanente.

il \(WIP\) è uguale al valore medio delle probabilitá ovvero:

\begin{equation} \label{eq:wipCtP}
    \begin{split}
        WIP &= \sum_{n = 0}^{\infty}(n*P_n)\\
        CT  &= \frac{WIP}{\lambda}\\
        P_1 &= \left(1-\frac{\lambda}{u}\right)\left(\frac{\lambda}{u}\right)^n
    \end{split}
\end{equation}

\begin{equation}\label{eq:wipdimostrazione}
    \begin{split}
        &\underbrace{\sum_{n = 0}^{\infty} P_n = \sum_{n = 0}^{\infty} \underbrace{n(1-u)u^n}_{P_n}}_{WIP}=\\
        &=(1-u)\sum_{n = 0}^{\infty} n*u^{n-1}*u=\\
        &=(1-u)u\sum_{n = 0}^{\infty} \underbrace{n*u^{n-1}}_{(A)=\frac{\partial (u^n)}{\partial n}} = \\
        &=(1-u)u\frac{\partial}{\partial u}\underbrace{\left(\sum_{n = 0}^{\infty} u^n \right)}_{\frac{1}{1-u}}=\\
        &=(1-u)u \frac{\partial}{\partial u} \left(\frac{1}{1-u}\right)=\\
        &=\frac{u}{1-u}=\\
        &=WIP
    \end{split}
\end{equation}

\begin{conditions*}
    (A) &   Conoscere la derivata è utile perché la somma delle derivate è uguale alla derivata della somma
\end{conditions*}

sostituendo nella \ref{eq:wipdimostrazione} \(u=\frac{\lambda}{\mu}\) abbiamo:
\begin{equation}
    \begin{split}
        WIP &= \frac{u}{1-u} =\\
        &= \frac{\frac{\lambda}{\mu}}{1-\frac{\lambda}{\mu}}=\\
        &= \frac{\lambda}{\mu - \lambda}
    \end{split}
\end{equation}

a sua volta si sostituisce nella \ref{eq:wipCtP}:
\begin{equation}\label{eq:ctconseguenza}
    \begin{split}
        CT &= \frac{WIP}{CT} =\\
        &=\frac{\lambda}{\mu-\lambda}*\frac{1}{\lambda}=\\
        &=\frac{1}{\mu-\lambda}
    \end{split}
\end{equation}

dalla \ref{eq:ctconseguenza} ne deriva che \((\mu-\lambda)>0 \implies \mu>\lambda\), nei capitoli successivi verrá illustrato come costruire la funzione di probabilitá.

Nota la probabilitá si può determinare il \(WIP\) perché solitamente il \(TH\) è noto

\subsubsection{Sistema a WIP controllato}

il \(WIP\) in questo caso è noto, le stazioni sono a singolo servente con il tempoo indicato sotto ciascuna stazione.

\begin{figure}[H]
    \centering
    \includegraphics[width=.7\linewidth]{images/chapter1/9.jpg}
    \caption{Sistema con WIP noto}
    \label{fig:sitemaWIPnoto}
\end{figure}

ricordiamo dalla \ref{eq:littleLaw} come ricavare \(WIP\) e \(TH\), il \(TH_{max}\) si conosce a prescindere in quanto rappresenta la capacitá della \(WK\) più lenta (collo di bottiglia), in questo caso è la seconda stazione con tempo pari a 2h:

\[
   TH_{max} =  \frac{1}{2}job/h = 0.5job/h
\]

\textbf{Tentativo 1: \(WIP=1\)}
\begin{figure}[H]
    \centering
    \includegraphics[width=.7\linewidth]{images/chapter1/10.jpg}
    \caption{Tentativo con \(WIP=1\)}
    \label{fig:wip1}
\end{figure}

Il sistema va immediatamente all'equilibrio, non si ha una struttura atipica iniziale.
\begin{align*}
    &TH(WIP=1)=\frac{1}{5}job/h\\
    &CT(WIP=1) = \frac{WIP}{TH} = \frac{1}{\frac{1}{5}} = 5h
\end{align*}

con \(WIP=1\) non raggiungo mai \(TH_{max}\).

\textbf{Tentativo 2: \(WIP=2\)}
\begin{figure}[H]
    \centering
    \includegraphics[width=.7\linewidth]{images/chapter1/11.jpg}
    \caption{Tentativo con \(WIP=2\)}
    \label{fig:wip2}
\end{figure}

nei primi 7 periodi escono 2 job, non è peró il ritmo di equilibrio, a \(\infty\) invece escono 2 job ogni 5 periodi perciò:
\begin{align*}
    &TH_{transitorio}(WIP=2)=\frac{2}{7}job/h\\
    &TH_{\infty}(WIP=2)=\frac{2}{5}job/h\\
    &TH(WIP=2)=\frac{2+2n}{7+5n}job/h\\
    &CT(WIP=2) = \frac{WIP}{TH} = \frac{2}{\frac{2}{5}} = 5h
\end{align*}

\textbf{Tentativo 3: \(WIP=3\)}
\begin{figure}[H]
    \centering
    \includegraphics[width=.7\linewidth]{images/chapter1/12.jpg}
    \caption{Tentativo con \(WIP=3\)}
    \label{fig:wip3}
\end{figure}

nei primi 9 periodi escono 3 job, non è peró il ritmo di equilibrio, a \(\infty\) invece escono 3 job ogni 6 periodi perciò:
\begin{align*}
    &TH_{transitorio}(WIP=3)=\frac{3}{9}\frac{1}{3}job/h\\
    &TH_{\infty}(WIP=3)=\frac{3}{4}=\frac{1}{2}job/h\\
    &TH(WIP=3)=\frac{3+3n}{9+6n}job/h\\
    &CT(WIP=3) = \frac{WIP}{TH} = \frac{3}{\frac{3}{6}} = 6h
\end{align*}
\begin{figure}[H]
    \centering
    \includegraphics[width=.7\linewidth]{images/chapter1/13.jpg}
    \caption{\(TH\) e \(CT\) con \(WIP=3\)}
    \label{fig:thctwip3}
\end{figure}

Per un sistema del genere la soluzione più efficiente si ottiene con \(WIP=3\) perché offre \(TH_{max}\), la condizione per raggiungere \(TH_{max}\) è pagare un piccolo aumento di \(CT\), se continuo ad aumentare \(CT\) il \(TH\) non aumenterá perchè è presente un collo di bottiglia.

La stazione 2 ha una media di 2h, il che significa che, presa singolarmente potrebbe avere durate diverse:

\begin{figure}[H]
    \centering
    \includegraphics[width=.7\linewidth]{images/chapter1/2h.jpg}
\end{figure}

È quindi necessario studiare i sistemi separatamente come se fossero deterministici, per farlo viene stabilito che:
\begin{align*}
    &TH_1=\alpha\\
    &TH_3=\beta
\end{align*}
Nella prima alternativa (1h) si ha:
\[TH_{u.c.}=0.5\alpha + 0.5\beta\]
Nella seconda alternativa (3h) si ha:
\[TH_{u.c.}=0.25\alpha + 0.75\beta\]

La soluzione è quindi la media pesata tra le due alternative:
\begin{figure}[H]
    \centering
    \includegraphics[width=.7\linewidth]{images/chapter1/14.jpg}
\end{figure}
È più corretto considerare il caso numero 2 perché si ha un comportamento \textit{veloce} una volta ogni quattro, quindi pesa meno.

\textbf{Tentativo 4: \(WIP=4\)}
Nel Primo si ha \(TH=\frac{2}{3}job/h\), nel secondo \(TH=\frac{1}{2}job/h\) perchè in questo caso il caso \(lento\) pesa di più e ci si ritrova un \(TH\) inferiore



