\newglossaryentry{CT}{
    name={\ensuremath{CT}},
    description={Cicle Time, tempo di permanenza di un job nel sistema}
}

\newglossaryentry{JOB}{
    name={\ensuremath{JOB}},
    description={Unitá da produrre, non è importante capire se si tratta di un prodotto singolo o più prodotti}
}

\newglossaryentry{tsi}{
    name={\ensuremath{T_s(i)}},
    description={Tempo di servizio o tempo di processo dello stadio \textit{i}, dato che è sempre affetto da variabilitá sará una \textit{variabile casuale}}
}

\newglossaryentry{etsi}{
    name={\ensuremath{E(T_s(i))}},
    description={Si tratta di una media perchè \gls{tsi} è una \textit{variabile casuale}, quindi si considera la media}
}

\newglossaryentry{cti}{
    name={\ensuremath{CT(i)}},
    description={ Cicle Time di un singolo stadio \textit{i}, dato che il tempo di sistema è una \textit{Variabile Casuale} allora anche il Cicle Time sará una \textit{variabile casuale}}
}

\newglossaryentry{cts}{
    name={\ensuremath{CT_s}},
    description={Cicle Time dell'intero sistema: è una funzione dei \gls{cti}, è la somma dei singoli \gls{cti} solo se si tratta di un sistema lineare}
}

\newglossaryentry{ctqi}{
    name={\ensuremath{CT_q(i)}},
    description={Tempo di attesa o Queue time
    \begin{figure}[H]
        \centering
        \includegraphics[width=.4\linewidth]{images/chapter1/3.jpg}
    \end{figure}}
}

\newglossaryentry{wip}{
    name={\ensuremath{WIP}},
    description={Misura il numero di Job presenti nel sistema in un determinato istante}
}

\newglossaryentry{wips}{
    name={\ensuremath{WIP_s}},
    description={Misura il numero di Job nel sistema in un determinato istante se si considera il sistema intero}
}

\newglossaryentry{wipi}{
    name={\ensuremath{WIP_i}},
    description={Misura il numero di Job nel sistema in un determinato istante se si considera lo stadio \textit{i}}
}

\newglossaryentry{th}{
    name={\ensuremath{TH}},
    description={Throughput, è il tasso di produzione del sistema, è legato alla velocitá della macchina, non è infatti l'inverso del tempo ma più correttamente il numero di Job che escono dal sistema nell'unitá di tempo (i.e. ogni giorno, ogni mese, ogni ora etc.)}
}

\newglossaryentry{tassoArrivo}{
    name={\ensuremath{\lambda}},
    description={Corrisponde spesso al \gls{th} del sistema ma vale per l'ingresso, rappresenta il numero di Job che entrano nel sistema nell'unitá di tempo, quando ho degli split e dei merge semplicemente sommo o divido per una certa percentuale, verrá praticamente sempre assunto che tutto ciò che entra nel sistema deve poi anche uscire quindi spesso \(\lambda\)=\gls{th}}
}

\newglossaryentry{tassoServizio}{
    name={\ensuremath{\mu}},
    description={Tasso di servizio di un servente: è la quantita massima di job che il server riesce a servire nell'unitá di tempo}
}

\newglossaryentry{utilizzo}{
    name={\ensuremath{u}},
    description={rappresenta l'utilizzo e vale \(\frac{\lambda}{\mu}\)}
}

\newglossaryentry{tempoInterarrivo}{
    name={\ensuremath{\frac{1}{\lambda}}},
    description={Tempo di interarrivo, dato che \gls{tassoArrivo} è un tasso, per trovare il tempo occorre calcolarne il reciproco}
}

\newglossaryentry{tempoServizio}{
    name={\ensuremath{\frac{1}{\mu}}},
    description={Tempo di servizio, dato che \gls{tassoServizio} è un tasso, per trovare il tempo occorre calcolarne il reciproco}
}

\newglossaryentry{workstation}{
    name={\ensuremath{WK}},
    description={Workstation / stazione / servente: il contenitore di uno o più servernti/macchine
    \begin{figure}[H]
        \centering
        \includegraphics[width=.4\linewidth]{images/chapter1/5.jpg}
    \end{figure}}
}

\newglossaryentry{step}{
    name={\ensuremath{STEP}},
    description={Coppia di workstation e tempo di processo di quella workstation. 
    \begin{itemize}
        \item step(M1,5min)
        \item step(M2,10min)
        \item step(M3,8min)
        \item step(M2,2min)
    \end{itemize}
    In questo caso ci sono 4 step ma le workstation sono 3 (M1, M2, M3), sono quindi presenti dei flussi rientranti, non c'è corrispondenza biunivoca tra workstation e step. \textit{N.b.} diversi Job sono dello stesso tipo solo se hanno gli stessi Step, ovvero se hanno lo stesso routing}
}

\newglossaryentry{routing}{
    name={\ensuremath{ROUTING}},
    description={Sequenza di step}
}

\newglossaryentry{Layout}{
    name={\ensuremath{LAYOUT}},
    description={Tutti i Job che passano nel sistema determinano il Layout dello stesso}
}

\newglossaryentry{At}{
    name={\ensuremath{A(t)}},
    description={Numero di Job entranti fino al tempo \textit{t}}
}

\newglossaryentry{Dt}{
    name={\ensuremath{A(t)}},
    description={Numero di Job usciti dal sistema fino al tempo \textit{t}}
}

\newglossaryentry{Pn}{
    name={\ensuremath{P_n}},
    description={\textit(Prob{N=n}) ovvero la probabilitá di avere un numero di job nel sistema pari a \textit{n}}
}